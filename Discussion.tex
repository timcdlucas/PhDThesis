
\tmpsection{What I did}

\begin{itemize}
\item In this thesis I aimed to test the importance of population structure and density on pathogen diversity
\item With a particular focus on bats
\item Combining simulations and empirical studies.
\item Identify that population structure does affect pathogen richness in wild bats.
\item But found that invasision of new pathogens is probably not the mechanism.  
\item Clarified the relationships between population size and density, range size, colony size and population structure.
\item And found that colony size is more important than density per se. 
\item Finally, created a method to more easily estimate bat population densities.
\end{itemize}



\tmpsection{How I did it (Chapters)}

\begin{enumerate}

\item
\begin{itemize}
\item I tested the hypothesis that population structure predicts viral richness in wild bats.
\item I used two measurements of population structure.
\begin{enumerate}
  \item A novel measure, number of subspecies. Largest dataset to date.
  \item Gene flow, dealing with issues of marker type and spatial scale
\end{enumerate}
\item I used multivariate regression, appraised with information theory techniques.
\item I controlled for phylogeny.
\item I found that in both analyses, increased population structure predicts increased pathogen richness and is in best model.
\end{itemize}

\item
\begin{itemize}
\item I modelled a multi-pathogen, metapopulation based on bat populations.
\item Testing the specific mechanism that population structure increases richness by enabling invasion of newly evolved pathogens.
\item I found that spatial structure, either by dispersal rate or topology, did not allow invasion.
\end{itemize}

\item
\begin{itemize}
\item I clarified confusion on the relationships between group size, group number, density, population size and range size.
\item Using same model as Chapter 3 I tested whether it is in fact density or population size that matters.
\item I tested whether the importact factor in increased density is group size or group number.
\item I found that it is in fact abundance and group size that matter much more than other factors.
\end{itemize}


\item
\begin{itemize}
\item I aimed to collect data on bat density as lit. says this is important.
\item Discovered I needed a model.
\item Started with specific model for iBats, ended up writing general model.
\item I formulated a model that estimates density from acoustic sensors or camera traps.
\item I tested it using simulations.
\item I found it to be precise and unbiased.
\end{itemize}

\end{enumerate}




\tmpsection{What was congruent with the literature?}  



\tmpsection{What was surprising?}

\begin{itemize}
\item Don't find high $R_0$ leads to high richness.
\item Find that colony size is more important that implying density is just proxy for group size.
\item 


\end{itemize}


\tmpsection{What are the limitations to the study?}



\tmpsection{What are the implications for practice?}


\begin{itemize}
\item Global change and pop structure.
\item Studies should more carefully consider density vs structure vs group size.
\item Can more easily estimate bat density.

\end{itemize}


\tmpsection{What are the implications for research?}



\tmpsection{Furtherwork}

\begin{itemize}
\item Examine more carefully the mechanisms for richness
\item Examine multi host species more carefully
\item Field test gREM
\end{itemize}


