\maketitle
%\makedeclaration

\begin{abstract} % 300 word limit

\lettr{T}he huge number of pathogen species strongly affects human health and ecological systems. 
I examine the role of host population structure and size in maintaining pathogen species richness in an important reservoir host for zoonotic viruses, bats (Order, Chiroptera). 
Firstly I test whether population structure is associated with high viral richness across wild bat species within a comparative phylogenetic analysis. 
I find evidence that bat species with more structured populations have more virus species. 
As this type of study cannot distinguish between specific mechanisms, I then formulate epidemiological models to test whether more structured host populations may allow invading pathogens to avoid competition. 
These models show that population structure does not affect the rate of pathogen invasion by this mechanism. 
Rather, in these models only the disease dynamics within the local group matter. 
As both global host population structure and local group size appear to be important for disease invasion, I use the same modelling framework to compare the importance of host group size and number of groups. 
I find that host group size has a stronger affect than number of groups. 
There are very few population size estimates for bats to directly test the importance of host population size on pathogen richness. 
Therefore I develop a method for estimating bat population sizes from acoustic surveys to assist future research. 
Overall in this thesis, I show that the structure and size of host bat populations can affect their ability to maintain many pathogen species and I provide a method to measure population sizes of bats. These findings increase our understanding of the ecological process of pathogen community construction and can help optimise host surveillance for zoonotic pathogens.

\end{abstract}





%%%%%%%%%%%%%%%%%%%%%%%%%%%%%%%%%%%%%%%%%%%%%%%%%%%%%%%%%%%%%%
%% Acknowledgements                                         %%
%%%%%%%%%%%%%%%%%%%%%%%%%%%%%%%%%%%%%%%%%%%%%%%%%%%%%%%%%%%%%%


\begin{acknowledgements}


\tmpsection{Kat + dylan, mum and dad}

Firstly and most importantly I would like to thank my wife, Katrina, for helping me beyond measure throughout my PhD and my son, Dylan, for making my life tiring and brilliant for the last two years.
You are 



\tmpsection{supervisors}

Secondly, I would like to thank my supervisors Kate and Hilde.


\tmpsection{Friends and that}


\tmpsection{Others?}


\end{acknowledgements}

\setcounter{tocdepth}{2} 
% Setting this higher means you get contents entries for
%  more minor section headers.

\tableofcontents
\listoffigures
\listoftables

